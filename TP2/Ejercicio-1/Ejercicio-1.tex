\documentclass[a4paper]{article}
\usepackage[utf8]{inputenc}
\usepackage[spanish, es-tabla, es-noshorthands]{babel}
\usepackage[table,xcdraw]{xcolor}
\usepackage[a4paper, footnotesep = 1cm, width=20cm, top=2.5cm, height=25cm, textwidth=18cm, textheight=25cm]{geometry}
%\geometry{showframe}

\usepackage{tikz}
\usepackage{amsmath}
\usepackage{amsfonts}
\usepackage{amssymb}
\usepackage{float}
\usepackage{graphicx}
\usepackage{caption}
\usepackage{subcaption}
\usepackage{multicol}
\usepackage{multirow}
\setlength{\doublerulesep}{\arrayrulewidth}
\usepackage{booktabs}

\usepackage{hyperref}
\hypersetup{
    colorlinks=true,
    linkcolor=blue,
    filecolor=magenta,      
    urlcolor=blue,
    citecolor=blue,    
}

\newcommand{\quotes}[1]{``#1''}
\usepackage{array}
\newcolumntype{C}[1]{>{\centering\let\newline\\\arraybackslash\hspace{0pt}}m{#1}}
\usepackage[american]{circuitikz}
\usetikzlibrary{calc}
\usepackage{fancyhdr}
\usepackage{units} 

\graphicspath{{../Ejercicio-1/}{../Ejercicio-2/}}

\pagestyle{fancy}
\fancyhf{}
\lhead{22.67 - Señales Aleatorias}
\rhead{Lambertucci, Londero B., Moriconi, Musich, Tolaba}
\rfoot{Página \thepage}

\begin{document}

\subsection{Introducción}

El siguiente ejercicio parte de un análisis sobre el proceso aleatorio presente en la página 138 del libro selecto por la cátedra. El ensamble del mismo se detalla a continuación.

\begin{center}
    $y_{1} = 6$ $y_{2} = 3sin(t)$  $y_{3} = -3sin(t)$ \\
    $y_{4} = 3cos(t)$ $y_{5} = -3cos(t)$  $y_{6} = -6$
\end{center}

Para el análisis sobre los valores pedidos es necesario generar múltiples muestras sobre el proceso, en los instantes de tiempo requeridos. En primer lugar se obtiene un número entero al azar entre 1 y 6. Esto constituye el experimento que determina qué función miembro del ensamble resulta. Al ser 6 funciones y equiprobables esta resulta la manera más sencilla. A partir de la determinacion de la función correspondiente se evalua en los valores.

\begin{center}
    $Y(\pi/2)$ $Y(\pi/4)$ $Y(\pi)$ $Y(2\pi)$
\end{center}


Los mismos se organizan como un vector. El procedimiento detallado luego se repite una gran cantidad de veces, obteniendo un arreglo de vectores conteniendo muestras del proceso.

A continuación se muestran los resultados para $X(\pi/2)$ para N experimentos realizados.

%AGREGAR IMAGEN DE MUESTRAS EN PI/2

El valor esperado del proceso teóricamente se obtiene de la siguiente forma


(Si no da paja agregar demo, sino poner directo da 0).

Se observa que esta no depende del tiempo, por lo cual en cualquier instante que se analize dara 0. Esto se correlaciona con los valores obtenidos experimentalmente. De las muestras se buscan las que corresponden a X(pi/2) y se busca el promedio entre ellas. Se puede observar también que el valor esperado experimental, se aproxima a 0 a medida que la cantidad de experimentos aumenta

%Poner grafico valoresperado en funcion de las muestras.






\end{document}