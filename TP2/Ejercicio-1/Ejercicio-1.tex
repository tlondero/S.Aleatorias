%\documentclass[a4paper]{article}
\usepackage[utf8]{inputenc}
\usepackage[spanish, es-tabla, es-noshorthands]{babel}
\usepackage[table,xcdraw]{xcolor}
\usepackage[a4paper, footnotesep = 1cm, width=20cm, top=2.5cm, height=25cm, textwidth=18cm, textheight=25cm]{geometry}
%\geometry{showframe}

\usepackage{tikz}
\usepackage{mathrsfs, amsmath}
\usepackage{amsfonts}
\usepackage{amssymb}
\usepackage{float}
\usepackage{graphicx}
\usepackage{caption}
\usepackage{subcaption}
\usepackage{multicol}
\usepackage{multirow}
\setlength{\doublerulesep}{\arrayrulewidth}
\usepackage{booktabs}

\usepackage{hyperref}
\hypersetup{
    colorlinks=true,
    linkcolor=blue,
    filecolor=magenta,      
    urlcolor=blue,
    citecolor=blue,    
}

\usepackage{array}
\newcolumntype{C}[1]{>{\centering\let\newline\\\arraybackslash\hspace{0pt}}m{#1}}
\usepackage[american]{circuitikz}
\usetikzlibrary{calc}
\usepackage{fancyhdr}
\usepackage{units} 

\graphicspath{./Imagenes}

%\pagestyle{fancy}
\fancyhf{}
\rfoot{Página \thepage}
%\begin{document}
\subsection{Introducción}

El siguiente ejercicio parte de un análisis sobre el proceso aleatorio presente en la página 138 del libro selecto por la cátedra. Se realizarán simulaciones de dicho proceso y se calcularán experimentalmente la media, la varianza, la autocorrelación y el coeficiente de autocorrelación para ciertos valores de t dados y se realizará una comparación con los valores teóricos.

\subsection{Valores teóricos}

El experimento que determina el proceso es la tirada de un dado no cargado y el ensamble del mismo se detalla a continuación:
\begin{equation} 
	\begin{split}
		 &y_{1(t)} = 6 \\
		 &y_{2(t)} = 3sin(t) \\
		 &y_{3(t)} = -3sin(t) \\
		 &y_{4(t)} = 3cos(t) \\
		 &y_{5(t)} = -3cos(t) \\
		 &y_{6(t)} = -6
	\end{split}
\end{equation}

El proceso es $Y_{(t)} = y_{i(t)}$ donde i indica el número obtenido en la tirada del dado.\\

El valor esperado teórico del proceso se obtiene de la siguiente forma:

\begin{equation*}
\begin{gathered}
	E\left[Y_{(t)}\right] = \sum_{i=1}^{6}\left( P(Y_{(t)} = y_{i(t)}) \times y_{i(t)}\right) 
\end{gathered}
\end{equation*}

Reemplazando las funciones muestra dadas y que la probabilidad $P(Y_{(t)} = y_{i(t)})= \frac{1}{6}$ $ \forall i $ obtenemos que:	

\begin{equation*}
\begin{gathered}
	E\left[Y_{(t)}\right] = 0$ $\forall t 
\end{gathered}
\end{equation*}

La varianza se obtiene como:

\begin{equation*}
\begin{gathered}
	Var^{2}_{(t)} = E\left[Y_{(t)}^{2}\right]- \left(E\left[Y_{(t)}\right]\right)^{2}  = \sum_{i=1}^{6}\left( P(Y_{(t)} = y_{i(t)}) \times y_{i(t)}^{2}\right) - 0^2 = 15 $   $ \forall t 
\end{gathered}
\end{equation*}

Donde ya se obtuvo que $E\left[Y_{(t)}\right] = 0$ y 

\begin{equation*}
\begin{gathered}
	E\left[Y_{(t)}^{2}\right] = \sum_{i=1}^{6}\left( P(Y_{(t)} = y_{i(t)}) \times y_{i(t)}^{2}\right) 
\end{gathered}
\end{equation*}

Este proceso tiene media y "varianza" constantes para todo instante t.\\

La autocorrelación para dos instantes t1 y t2 se encuentra calculada en el libro y nos queda como:
\begin{equation*}
\begin{gathered}
	R_{xx(t_1,t_2)} = \frac{1}{6}\left(72+ 18 \cos(t_2 - t_1)\right)
\end{gathered}
\end{equation*}

El proceso tiene media constante y autocorrelación dependiente de $(t_2 - t_1)$, entonces es WSS. Por lo tanto:
\begin{equation*}
\begin{gathered}
	R_{xx(t,t)} = R_{xx(0,0)} = \frac{1}{6}\left(72+ 18 \cos(0)\right) = 15
\end{gathered}
\end{equation*}

El coeficiente de autocorrelación se obtiene con la definición del mismo:
\begin{equation*}
\begin{gathered}
	r_{xx(t_1,t_2)} = \frac{R_{xx(t_1,t_2)}-\mu_{X_{(t_1)}}^{*} \mu_{X_{(t_2)}}}{\left(R_{xx(t_1,t_1)}.R_{xx(t_2,t_2)}\right)^{1/2}} 
\end{gathered}
\end{equation*}

Como el proceso es WSS y su media es cero cualquiera sea t:

\begin{equation*}
\begin{gathered}
	r_{xx(t_1,t_2)} = \frac{R_{xx(t_1,t_2)}-0}{(R_{xx(0,0)}^2)^{1/2}} = \frac{\frac{1}{6}\left(72+ 18 \cos(t_2 - t_1)\right)}{15}
\end{gathered}
\end{equation*}


Para los instantes de t requeridos, obtenemos los siguientes resultados:
\begin{enumerate}
	\item[•]E$\left[ Y_{(\frac{\pi}{2})}\right]$= 0 
	\item[•]Var$\left[Y_{(\frac{\pi}{2})}\right]$= 15
	\item[•]$R_{xx(\frac{\pi}{4},\frac{\pi}{2})}$= $\frac{1}{6} (72 + 18 \cos(\frac{\pi}{4}))$ = 14.12132
	\item[•]$r_{xx(2\pi,\pi)} = \frac{R_{xx(\pi,2\pi)}}{15} = \frac{ (12 + 3 \cos(\pi))}{15}$ = 0.6 \\
\end{enumerate}

Observando el ensamble dado se puede comprobar fácilmente que el proceso no es ergódico en la media puesto que con la función muestra $y_{1(t)} = 6$:

\begin{equation*}
\begin{split}
	\lim_{T\to\infty} < Y_{(t)} >_T = & \lim_{T\to\infty} \frac{1}{T} \int_{-T/2}^{T/2} Y(t) dt = 1 \neq \mu 
\end{split}
\end{equation*}

De la misma forma, se puede concluir que no es ergódico en la autocorrelación con la misma función muestra:

\begin{equation*}
\begin{split}
	\lim_{T\to\infty} < R_{YY}(\tau) >_T = & \lim_{T\to\infty} \frac{1}{T} \int_{-T/2}^{T/2} Y(t) Y(t + \tau) dt = 6 \neq R{YY}(\tau) \ 	\forall \tau
\end{split}	
\end{equation*}
%\underbrace{\frac{1}{6}}{y_1(t)}
Dado que para una de las funciones no se cumple que tienda a la autocorrelación, no es ergódico en en dicha variable.\\


\subsection{Análisis Experimental}

Para el análisis sobre los valores pedidos es necesario generar múltiples muestras sobre el proceso, en los instantes de tiempo requeridos. En primer lugar, se obtiene un número entero al azar entre 1 y 6, simulando la tirada de un dado, el cual determina qué función miembro del ensamble resulta.
A partir de la determinación de la función correspondiente se evaluá en los valores de instantes t pedidos, obteniéndose:

\begin{enumerate}
   \item[•] $Y(\pi/2)$
   \item[•] $Y(\pi/4)$
   \item[•] $Y(\pi)$
   \item[•] $Y(2\pi)$
\end{enumerate}

Estos valores obtenidos se guardan como un vector. Luego, se repite el procedimiento N = 1000 veces 
y se obtiene un arreglo de vectores conteniendo muestras del proceso.
El código de Matlab empleado para la simulación de este proceso se detalla a continuación 
\\


\lstinputlisting[language=Matlab]{../Ejercicio-1/Matlab/simulacion.m}
%\begin{figure}[H]
%\centering
%	\includegraphics[width=0.8\textwidth, trim = {0 0 0 0},clip]{./ImagenesEjercicio1/main1.png}
%	\caption{Código Matlab de la simulación del proceso.}
%	\label{fig:main1}
%\end{figure}

Donde la función fun-array() designa el ensamble solicitado, el código en Matlab:

\lstinputlisting[language=Matlab]{../Ejercicio-1/Matlab/fun_array.m}
%\begin{figure}[H]
%\centering
%	\includegraphics[width=0.6\textwidth, trim = {0 0 0 0},clip]{./ImagenesEjercicio1/fun_array.png}
%	\caption{La función que contiene el ensamble del proceso.}
%	\label{fig:fun_array}
%\end{figure}

A continuación, por ejemplo se muestran los resultados para $Y(\pi/2)$ para N = 100 experimentos realizados.

\begin{figure}[H]
\centering
	\includegraphics[width=0.5\textwidth, trim = {0 0 0 0},clip]{./ImagenesEjercicio1/ypi_2.png}
	\caption{Valores del proceso $Y_(t)$ en $t = \frac{\pi}{2} $.}
	\label{fig:ypi_2}
\end{figure}

También para $Y(\pi/4)$.

\begin{figure}[H]
\centering
	\includegraphics[width=0.5\textwidth, trim = {0 0 0 0},clip]{./ImagenesEjercicio1/ypi_4.png}
	\caption{Valores del proceso $Y_(t)$ en $t = \frac{\pi}{4} $.}
	\label{fig:ypi_4}
\end{figure}

Luego, se calculan promediando los valores pedidos con el código:

\begin{lstlisting}
%Ploteo valores de funcion evaluada en t = pi/2 para multiples experimentos
figure (1);
ejex = linspace(1,cantidad_muestras,cantidad_muestras);
scatter(ejex, muestras_totales(1,:));

%Ploteo valores de funcion evaluada en t = pi/4 para multiples experimentos
figure(2);
ejex = linspace(1,cantidad_muestras,cantidad_muestras);
scatter(ejex, muestras_totales(2,:));

%Estimamos la media en t1= pi/2
exp_mean_t1 = expected_value(cantidad_muestras,muestras_totales(1,:));

%Estimamos la varianza en t2= pi/2
var_t1 = var_exp(cantidad_muestras,muestras_totales(1,:));

%Estimamos la autocorrelacion en t1= pi/2 y t2= pi/4
autocorr_t1_t2 = autocorr_exp(cantidad_muestras,muestras_totales(1,:),muestras_totales(2,:));

%Estimamos el coeficiente de autocorrelacion en t3= pi y t4= 2pi
coef_autocorr_t3_t4 = autocorr_coef_exp(cantidad_muestras,muestras_totales(3,:),muestras_totales(4,:));
\end{lstlisting}
%\begin{figure}[H]
%\centering
%	\includegraphics[width=0.8\textwidth, trim = {0 0 0 0},clip]{./ImagenesEjercicio1/main2.png}
%	\caption{Código de Matlab de la simulación del proceso.}
%	\label{fig:main2}
%\end{figure}

Detallando cada función:
\begin{enumerate}
\item[•] La función estimadora de la media en $t = \frac{\pi}{2}$, E$\left[ Y_{(\frac{\pi}{2})}\right]$
\lstinputlisting[language=Matlab]{../Ejercicio-1/Matlab/expected_value.m}
%\begin{figure}[H]
%\centering
%	\includegraphics[width=0.6\textwidth, trim = {0 0 0 0},clip]{./ImagenesEjercicio1/expval.png}
%	\caption{La función que calcula la media experimental del proceso.}
%	\label{fig:expval}
%\end{figure}

\item[•] La función estimadora de la varianza en $t = \frac{\pi}{2}$, Var$\left[Y_{(\frac{\pi}{2})}\right]$
\lstinputlisting[language=Matlab]{../Ejercicio-1/Matlab/var_exp.m}
%\begin{figure}[H]
%\centering
%	\includegraphics[width=0.6\textwidth, trim = {0 0 0 0},clip]{./ImagenesEjercicio1/expvar.png}
%	\caption{La función que calcula la varianza experimental.}
%	\label{fig:expvar}
%\end{figure}

\item[•] La función estimadora de la autocorrelación en $t_1 = \frac{\pi}{4}$ y $t_2 = \frac{\pi}{2}$, $R_{xx(\frac{\pi}{4},\frac{\pi}{2})}$
\lstinputlisting[language=Matlab]{../Ejercicio-1/Matlab/autocorr_exp.m}
%\begin{figure}[H]
%\centering
%	\includegraphics[width=0.6\textwidth, trim = {0 0 0 0},clip]{./ImagenesEjercicio1/autocorr.png}
%	\caption{La función estimadora de la autocorrelación.}
%	\label{fig:autocorr}
%\end{figure}

\item[•] La función estimadora del coeficiente de autocorrelación en $t_3 = \frac{2\pi}{4}$ y $t_4 = \pi$, $r_{xx(2\pi,\pi)}$
\lstinputlisting[language=Matlab]{../Ejercicio-1/Matlab/autocorr_coef_exp.m}
%\begin{figure}[H]
%\centering
%	\includegraphics[width=0.6\textwidth, trim = {0 0 0 0},clip]{./ImagenesEjercicio1/coefauto.png}
%	\caption{La función estimadora del coeficiente de autocorrelación.}
%	\label{fig:coefauto}
%\end{figure}
\end{enumerate}

Corriendo la simulación para N = 1000, se arrojaron los siguientes resultados:
\begin{figure}[H]
\centering
	\includegraphics[width=0.6\textwidth, trim = {0 0 0 0},clip]{./ImagenesEjercicio1/result.png}
	\caption{Resultados de la simulación con N = 1000 muestras}
	\label{fig:result}
\end{figure}

Observando el código anterior obtenemos que:
\begin{enumerate}
	\item[•]E$\left[ Y_{(\frac{\pi}{2})}\right]$= 0.0120 
	\item[•]Var$\left[Y_{(\frac{\pi}{2})}\right]$= 14.6159
	\item[•]$R_{xx(\frac{\pi}{4},\frac{\pi}{2})}$= 13.7303
	\item[•]$r_{xx(2\pi,\pi)}$= 0.5801
\end{enumerate}

Adicionalmente, analizamos para la media en $t_1 = \frac{\pi}{2}$ que E$\left[ Y_{(\frac{\pi}{2})}\right]\rightarrow 0$ a medida que se realizan simulaciones con N$\rightarrow \infty$
\begin{figure}[H]
\centering
	\includegraphics[width=0.6\textwidth, trim = {0 0 0 0},clip]{./ImagenesEjercicio1/media.png}
	\caption{El valor esperado del proceso en $t_1 = \frac{\pi}{2}$ cuando la cantidad de muestras aumenta.}
	\label{fig:media}
\end{figure}

\subsection{Conclusiones sobre los resultados}

En primer lugar, en las figuras (\ref{fig:ypi_2}) y (\ref{fig:ypi_4}) se puede "estimar" visualmente que la media para el proceso es cero como primer aproximación a los valores teóricos.

Luego, se puede concluir que para los valores experimentales, la media en $t_1 = \frac{\pi}{2}$ es cercana a cero y a medida que aumentamos la cantidad de valores muestreados la diferencia entre el valor experimental y el valor teórico es cada vez más pequeña. De igual manera, para la varianza en $t_1 = \frac{\pi}{2}$, no se observan diferencias significativas.
También, la autocorrelación en $t_1 = \frac{\pi}{2}$ y $t_2 = \frac{\pi}{4}$ y la autocorrelación en $t_3 = \pi$ y $t_4 = 2\pi$ denotan el mismo comportamiento hacia el valor teórico. 

A medida que se toman mayor cantidad de muestras $(N \rightarrow \infty )$ los valores que se estimaron convergieron a los valores teóricos, lo cual era de esperarse puesto que el proceso tiene un ensamble simétrico y equiprobable. 

Los valores estimados con los muestreos del proceso no pueden estimarse mediante promedios temporales eligiendo alguna de las funciones muestras experimentales porque el proceso no es ergódico en la media ni tampoco en la autocorrelación. Esto puede observarse fácilmente cuando el experimento aleatorio que determina el proceso cae en los valores de $y_{1(t)} = 6$ o $y_{6(t)} = -6$.

%\end{document}
