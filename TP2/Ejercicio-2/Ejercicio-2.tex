\documentclass[a4paper]{article}
\usepackage[utf8]{inputenc}
\usepackage[spanish, es-tabla, es-noshorthands]{babel}
\usepackage[table,xcdraw]{xcolor}
\usepackage[a4paper, footnotesep = 1cm, width=20cm, top=2.5cm, height=25cm, textwidth=18cm, textheight=25cm]{geometry}
%\geometry{showframe}

\usepackage{tikz}
\usepackage{mathrsfs, amsmath}
\usepackage{amsfonts}
\usepackage{amssymb}
\usepackage{float}
\usepackage{graphicx}
\usepackage{caption}
\usepackage{subcaption}
\usepackage{multicol}
\usepackage{multirow}
\setlength{\doublerulesep}{\arrayrulewidth}
\usepackage{booktabs}

\usepackage{hyperref}
\hypersetup{
    colorlinks=true,
    linkcolor=blue,
    filecolor=magenta,      
    urlcolor=blue,
    citecolor=blue,    
}

\usepackage{array}
\newcolumntype{C}[1]{>{\centering\let\newline\\\arraybackslash\hspace{0pt}}m{#1}}
\usepackage[american]{circuitikz}
\usetikzlibrary{calc}
\usepackage{fancyhdr}
\usepackage{units} 

\graphicspath{./Imagenes}

%\pagestyle{fancy}
\fancyhf{}
\rfoot{Página \thepage}

\begin{document}

\subsection{Introducción}

Se analiza una secuencia $X(n)$, estimando y calculando parámetros de interés, como lo son la autocorrelación, los coeficientes de correlación parcial y la densidad espectral de potencia.

\subsection{Autocorrelación} 

Se estiman la autocorrelación mediante el uso de los primeros $128$ elementos de la secuencia brindada. Para ello, se vale los estimadores polarizados ($R_{p}$) y no polarizados ($R_{np}$) de dicho parámetro. Estas funciones son las empleadas para estimar otras funciones mediante información digitalizada.
\begin{equation*}
\begin{gathered}
	R_{p}(k) = \frac{1}{N} \sum_{i=0}^{N-k-1} X(i)X(i+k)	\\
	R_{np}(k) = \frac{1}{N-k} \sum_{i=0}^{N-k-1} X(i)X(i+k)
\end{gathered}
\end{equation*}
En ellas se observan los parámetros $N$, es decir, el largo de $X(n)$, y $k$, variable que puede tomar los valores $0, 1, ... \ , 127$. Mediante el uso de estos estimadores, se normaliza para poder obtener los coeficientes de autocorrelación $r_{XXp}$ y $r_{XXnp}$. 

\begin{figure}[H]
\centering
	\includegraphics[width=0.6\textwidth, trim = {0 0 0 0.7cm},clip]{./ImagenesEjercicio2/rxx.png}
	\caption{Grafica de los coeficientes de autocorrelación total estiamdos.}
	\label{fig:rxx}
\end{figure}

Se puede observar en la Figura (\ref{fig:rxx}) como ambas curvas se encuentran solapadas, haciendo que sea prácticamente imposible distinguirlas.
Esto se debe a que existe una relación entre cada estimador, siendo esta
\begin{equation*}
\begin{gathered}
	R_{p}(k) = \frac{N - k}{N} R_{np}(k)
\end{gathered}
\end{equation*}

Ya que, para el caso del vector analizado, se da la condición de que $N = 4096$ y además $N >> k_{max} = 127$, siendo entonces
\begin{equation*}
\begin{gathered}
	R_{p}(k) \approx R_{np}(k)
\end{gathered}
\end{equation*}

\subsection{Coeficientes de correlación parcial}

Con los datos ya extraídos y mediante la resolución de la ecuación de Yule-Walker, fue posible obtener los coeficientes deseados. Esto se realizó con los coeficientes totales obtenidos a través de las estimaciones polarizada y no polarizada.
\begin{figure}[H]
\centering
	\includegraphics[width=0.6\textwidth, trim = {0 0 0 0.7cm},clip]{./ImagenesEjercicio2/phikk.png}
	\caption{Grafica de los coeficientes de autocorrelación parcial obtenidos.}
	\label{fig:phikk}
\end{figure}

En la Figura (\ref{fig:phikk}), a diferencia de la anterior, se obtuvo una mayor diferencia entre ambas curvas, pero ser un cambio significativo \textcolor{red}{ACÁ SE PODRÍA PONER ALGO A MODO DE REFLEXIÓN, DE PORQUÉ PASA ESTO O DE QUE SIGNIFICA, QUE SE YO AUXILIO.}

\subsection{Acá vendría el punto 3 y 4 pero Frenkie me va a ayudar a escribirlo jaja beso}

\subsection{Densidad espectral de potencia}

A continuación, se estima la la densidad espectral de potencia del vector $X(n)$. Para ello, se emplean dos técnicas distintas. La primera consiste en el uso de la transformada de Fourier de la estimación realizada de las funciones de autocorrelación.
\begin{figure}[H]
\centering
	\includegraphics[width=0.6\textwidth, trim = {0 0 0 0.725cm},clip]{./ImagenesEjercicio2/period-est.png}
	\caption{Periodigramas obtenidos a partir de las estimaciones de $R_{XX}$.}
	\label{fig:period-est}
\end{figure}

Como era de esperarse, la diferencia entre el gráfico obtenido a través de la estimación polarizada no difiere tanto de la no polarizada.

La segunda técnica consta de la promediación de periodigramas. Transformando con Fourier el vector analizado se obtiene
\begin{figure}[H]
\centering
	\includegraphics[width=0.6\textwidth, trim = {0 0 0 0.725cm},clip]{./ImagenesEjercicio2/period-calc.png}
	\caption{Periodigrama calculado.}
	\label{fig:period-calc}
\end{figure}

\textcolor{red}{REFLEXIÓN DE LOS GRÁFICOS Y PORQUÉ UNO ES 5 VECES MÁS GRANDE QUE EL OTRO KCYO X2.}


\end{document}